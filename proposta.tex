\documentclass[12pt, a4paper]{article}
\usepackage{amsmath}
\usepackage{caption}
\usepackage{float}
\usepackage{array}
\usepackage[numbers]{natbib}
\usepackage{multirow}
\usepackage{amsfonts}
\usepackage{titlesec}
\usepackage{graphicx}
\usepackage[portuguese]{babel}
\usepackage[left=2cm, right=2cm, top=1.5cm]{geometry}

\begin{document}
\title{Desenvolvimento de biomarcadores a partir de voz: análise de áudio para previsão do nível de SpO2}

\author{Alunos\\ Natália Hitomi Koza \\ Ricardo Mikio Morita \\ \\Orientador\\ Prof. Dr. Marcelo Finger}
\maketitle

\section{Motivação}

Motivado pelo quadro de hipóxia silenciosa presente no COVID-19 foi realizado um estudo anterior \citep{casanova2021deep}, que apresentou bons resultados, com acurácia de 91,66\%, acerca da possibilidade de se classificar através da voz pacientes infectados ou não. No presente trabalho buscamos extender esse estudo, para outras doenças respiratórias através da detecção da insuficiência respiratória prevendo-se o nível de saturação de oxigênio no sangue (SpO2).


\section{Objetivos}

Estimar o nível de SpO2 através da análise de um pequeno trecho de fala, com um intervalo de confiança, para tentar detectar uma possível insuficiência respiratória, fazendo uso de redes neurais e dispositivos acessíveis e de baixo custo, como smartphones e computadores, para a coleta do áudio. E, futuramente, poder auxiliar na triagem hospitalar ou mesmo o público geral, na avaliação da necessidade de intervenção médica.

\section{Metodologia}

Inicialmente utilizaremos dados já coletados do ano anterior no Hospital das Clínicas, que são exclusivamente de pacientes com COVID-19 e pacientes saudáveis. Sendo que parte desses dados foi coletada na enfermaria, onde obteve-se a frequência cardíaca e saturação de oxigênio através do oxímetro e coletou-se as vozes com um celular, e grande parte dos dados via doação de voz para um projeto numa interface web, onde assumiu-se ser de indivíduos saudáveis.

Posteriormente pretendemos utilizar os novos dados, que estão sendo coletados em diversos hospitais, e que incluem outras doenças respiratórias.

Os sinais de áudio passarão por pré-processamento, como filtros, MFCC e geração de espectrograma, em seguida utilizaremos diferentes modelos de redes neurais em busca de encontrar uma com alta acurácia em identificar as doenças respiratórias.


\section{Planejamento}

\begin{enumerate}
    \item Estudo de processamento de sinais
    \item Estudo de redes neurais
    \item Estudo de artigos relacionados
    \item Treinamento e implementação de modelos
    \item Refinamento do modelo escolhido
    \item Análise dos resultados obtidos
    \item Escrever a monografia
    \item Preparar apresentação/pôster
\end{enumerate}

\begin{tabular}{|c|cccccccccc|}
    \hline
      & \multicolumn{10}{c}{Meses}\\
     Atividade & Mar & Abr & Maio & Jun & Jul & Ago & Set & Out & Nov & Dez  \\
     \hline
     1 & \checkmark & \checkmark & \checkmark &  &  &  &  &  &  &   \\
     2 & \checkmark & \checkmark & \checkmark &  &  &  &  &  &  &  \\
     3 & \checkmark & \checkmark & \checkmark & \checkmark & \checkmark & \checkmark & \checkmark &  &  &   \\
     4 &  &  & \checkmark & \checkmark & \checkmark & \checkmark &  &  &  &   \\
     5 &  &  &  &  &  & \checkmark & \checkmark &  &  &  \\
     6 &  &  &  &  &  &  & \checkmark & \checkmark &  &   \\
     7 &  &  &  &  &  &  &  & \checkmark & \checkmark & \checkmark  \\
     8 &  &  &  &  &  &  &  &  & \checkmark & \checkmark \\
    \hline
    \end{tabular}

\bibliographystyle{plainnat}
\bibliography{proposta}

\end{document}
