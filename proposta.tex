\documentclass[12pt, a4paper]{article}
\usepackage{amsmath}
\usepackage{caption}
\usepackage{float}
\usepackage{array}
\usepackage{multirow}
\usepackage{amsfonts}
\usepackage{titlesec}
\usepackage{graphicx}
\usepackage[portuguese]{babel}
\usepackage[left=2cm, right=2cm, top=1.5cm]{geometry}

\begin{document}
\title{Desenvolvimento de biomarcadores a partir de voz: análise de áudio para detecção do estado de saúde de pacientes admitidos no Hospital das Clínicas}

\author{Alunos: Natália Hitomi Koza e Ricardo Mikio Morita \\ Orientador: Prof. Marcelo Finger}
\maketitle

\section{Introdução}

\section{Motivação}

Motivado pelo quadro de hipóxia silenciosa presente no COVID-19 foi realizado um estudo anterior, que apresentou bons resultados, acerca da possibilidade de se identificar pacientes infectados através da voz. No presente trabalho buscamos extender esse estudo, para outras doenças respiratórias.


\section{Objetivos}

Estimar o nível de SpO2 através da análise de um pequeno trecho de fala, para tentar detectar possíveis problemas respiratórios. E, futuramente, poder auxiliar médicos ou mesmo o público geral, oferecendo um sistema portátil e de baixo custo.

\section{Metodologia}

\section{Planejamento}

\begin{enumerate}
    \item Estudo de processamento de sinais
    \item Estudo de redes neurais
    \item Estudo de artigos relacionados
    \item Treinamento e implementação de modelos
    \item Refinamento do modelo escolhido
    \item Análise dos resultados obtidos
    \item Escrever a monografia
    \item Preparar apresentação/pôster
\end{enumerate}

\begin{tabular}{|c|cccccccccc|}
    \hline
      & \multicolumn{10}{c}{Meses}\\
     Atividade & Mar & Abr & Maio & Jun & Jul & Ago & Set & Out & Nov & Dez  \\
     \hline
     1 & \checkmark & \checkmark & \checkmark &  &  &  &  &  &  &   \\
     2 & \checkmark & \checkmark & \checkmark &  &  &  &  &  &  &  \\
     3 & \checkmark & \checkmark & \checkmark & \checkmark & \checkmark & \checkmark & \checkmark &  &  &   \\
     4 &  &  & \checkmark & \checkmark & \checkmark & \checkmark &  &  &  &   \\
     5 &  &  &  &  &  & \checkmark & \checkmark &  &  &  \\
     6 &  &  &  &  &  &  & \checkmark & \checkmark &  &   \\
     7 &  &  &  &  &  &  &  & \checkmark & \checkmark & \checkmark  \\
     8 &  &  &  &  &  &  &  &  & \checkmark & \checkmark \\
    \hline
    \end{tabular}

\section*{Referências}

\end{document}
