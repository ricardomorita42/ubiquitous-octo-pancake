\documentclass[12pt, a4paper]{article}
\usepackage{amsmath}
\usepackage{caption}
\usepackage{float}
\usepackage{array}
\usepackage[numbers]{natbib}
\usepackage{multirow}
\usepackage{amsfonts}
\usepackage{titlesec}
\usepackage{graphicx}
\usepackage[portuguese]{babel}
\usepackage[left=2cm, right=2cm, top=1.5cm]{geometry}

\begin{document}
\title{Desenvolvimento de biomarcadores a partir de voz: análise de áudio para previsão do nível de SpO2}

\author{Alunos\\ Natália Hitomi Koza \\ Ricardo Mikio Morita \\ \\Orientador\\ Prof. Dr. Marcelo Finger}
\maketitle

\section{Motivação}

Insuficiência respiratória é um sintoma relevante ao determina se um paciente precisa de hospitalização. Durante as ondas da pandemia de COVID-19 que ocorreu em meados de 2020, uma das complicações ocorridas envolviam uma queda da saturação de oxigênio no sangue (SpO2) para abaixo de $95\%$, a qual é considerada um fator de risco e que necessita de atenção médica\cite{manualcovid}. Um estudo anterior realizado pelo grupo SPIRA foi motivado por este quadro de hipóxia silenciosa \citep{casanova2021deep}, levando ao estudo e criação de um biomarcador que apresentou bons resultados acerca da possibilidade de se identificar pacientes abaixo deste limiar com uma acurácia de $91.66\%$, com perspectivas de melhorar este número. No presente trabalho buscamos trabalhar numa linha similar, visando estimar os níveis de SpO2 com o uso estes biomarcadores além de tentar estimar uma aproximação da acurácia desta previsão.


\section{Objetivos}

Estimar o nível de SpO2 através da análise de um pequeno trecho de fala, com um intervalo de confiança, para tentar detectar uma possível insuficiência respiratória, fazendo uso de redes neurais e dispositivos acessíveis e de baixo custo, como smartphones e computadores, para a coleta do áudio. Espera-se obter uma ferramenta que possa ser útil, tanto pela praticidade ou quanto pelo baixo custo, na área médica no futuro.

\section{Metodologia}

Inicialmente utilizaremos dados coletados no estudo anterior, que são exclusivamente de pacientes com COVID-19 e pacientes saudáveis. Parte desses dados foi coletada na enfermaria, onde obteve-se a frequência cardíaca e saturação de oxigênio através do oxímetro e coletou-se as vozes com um celular. Uma grande parte dos dados foi obtida via doação de voz para o projeto numa interface web, onde assumiu-se ser de indivíduos saudáveis.

Posteriormente pretendemos utilizar os novos dados, que estão sendo coletados em diversos hospitais, e que incluem outras doenças respiratórias.

Os sinais de áudio passarão por pré-processamento, como filtros, MFCC e geração de espectrograma, em seguida utilizaremos diferentes modelos de redes neurais em busca de encontrar uma com alta acurácia em identificar as doenças respiratórias.

\section{Planejamento}

\begin{enumerate}
    \item Estudo de processamento de sinais
    \item Estudo de redes neurais
    \item Estudo de artigos relacionados
    \item Treinamento e implementação de modelos
    \item Refinamento do modelo escolhido
    \item Análise dos resultados obtidos
    \item Escrever a monografia
    \item Preparar apresentação/pôster
\end{enumerate}

\begin{tabular}{|c|cccccccccc|}
    \hline
      & \multicolumn{10}{c}{Meses}\\
     Atividade & Mar & Abr & Maio & Jun & Jul & Ago & Set & Out & Nov & Dez  \\
     \hline
     1 & \checkmark & \checkmark & \checkmark &  &  &  &  &  &  &   \\
     2 & \checkmark & \checkmark & \checkmark &  &  &  &  &  &  &  \\
     3 & \checkmark & \checkmark & \checkmark & \checkmark & \checkmark & \checkmark & \checkmark &  &  &   \\
     4 &  &  & \checkmark & \checkmark & \checkmark & \checkmark &  &  &  &   \\
     5 &  &  &  &  &  & \checkmark & \checkmark &  &  &  \\
     6 &  &  &  &  &  &  & \checkmark & \checkmark &  &   \\
     7 &  &  &  &  &  &  &  & \checkmark & \checkmark & \checkmark  \\
     8 &  &  &  &  &  &  &  &  & \checkmark & \checkmark \\
    \hline
    \end{tabular}

\bibliographystyle{plainnat}
\bibliography{proposta}

\end{document}
